\documentclass{article}
\usepackage{amssymb, amsmath, amsfonts, amsthm, graphicx, hyperref, stmaryrd, enumitem}
\hypersetup{
    colorlinks=true,
    linkcolor=blue,
    filecolor=magenta,
    urlcolor=cyan,
    }

{\theoremstyle{definition}
\newtheorem{problem}{Problem}[section]}

\newcommand{\pf}{\rightharpoonup}
\newcommand{\f}{\rightarrow}
\newcommand{\ev}[1]{\Downarrow_\text{#1}}
\newcommand{\evt}[1]{\Downarrow_\textit{#1}}
\newcommand{\den}[1]{\llbracket #1 \rrbracket}

\title{Denotational Semantics and Domain Theory}
\author{}
\date{}

\begin{document}
\maketitle

\noindent
I am working through
\href{https://www.cl.cam.ac.uk/teaching/1112/DenotSem/dens-notes-bw.pdf}{these
Cambridge notes}. The problem numbers don't line up. Sue me.

\section{Basics of Denotational Semantics (from Cambridge notes)}
\begin{problem}
    Consider the function
    \[
        f_{b,c} : (State \pf State) \f (State \pf State)
    \]
    defined on Slide 11.
    \begin{enumerate}[label=\alph*)]
        \item Show by induction on $n$ that
            \[
                f_{b,c}^n(\bot)=\lambda s\in State.\begin{cases}
                    c^k(s) & \text{if $0\leq k <n$ is such that $b(c^i(s))=true$} \\
                    & \text{for all $0\leq i<k$ and $b(c^k(s))=false$} \\[2ex]
                    \text{undefined} & \text{if $b(c^i(s))=true$ for all $i\geq 0$}
                \end{cases}
            \]
        \item Let $w_{b,c}:State \pf State$ be the partial function defined
            as
            \[
                w_{b,c}\stackrel{\text{def}}{=} \lambda s\in
                State.\begin{cases}
                    c^k(s) & \text{if $k\geq 0$ is such that $b(c^i(s))=true$} \\
                    & \text{for all $0\leq i<k$ and $b(c^k(s))=false$} \\[2ex]
                    \text{undefined} & \text{if $b(c^i(s))=true$ for all $i\geq 0$}
                \end{cases}
            \]
            Show that $w_{b,c}$ satisfies the fixed-point equation
            \[
                w_{b,c}=f_{b,c}(w_{b,c}).
            \]
        \item Describe the function $f_{b,c}$ for
            $b=\llbracket\textbf{true}\rrbracket=\lambda s\in State.true$ and
            $c=\llbracket \textbf{skip}\rrbracket=\lambda s\in State.s$. Which
            partial functions from states to states are fixed points of this
            $f_{b,c}$? What is its least fixed point (with respect to the
            $\sqsubseteq$ ordering defined above)? Does this least fixed point
            agree with the partial function from states to states determined by
            the operational semantics of \textbf{while true do skip}?
    \end{enumerate}
\end{problem}
\begin{proof}[Solution]
    .
    \begin{enumerate}[label=\alph*)]
        \item In the base case, let $n=1$ and we have $f_{b,c}(\bot)=\lambda
            s\in State. \textit{ if}\,(b(s),\bot,s)$. This gives us the
            definition
            \[
                f_{b,c}(\bot)=\lambda s\in State.\begin{cases}
                    \text{undefined} & \text{if $b(c^0(s))=true$} \\[2ex]
                    c^0(s) & \text{if $0\leq k<1$ is such that $b(c^i(s))=true$} \\
                    & \text{for all $0\leq i<k$ and $b(c^0(s))=false$}
                \end{cases}
            \]
            where $g^0(s)=s$ for any function $g$. For the inductive case,
            suppose that we have the function $f_{b,c}^n(\bot)$ as defined in
            the problem statement and we wish to compute
            $f_{b,c}(f_{b,c}^n(\bot))$. By the definition, we get the function
            \[
                f_{b,c}(f_{b,c}^n(\bot)) = \lambda s\in State. \begin{cases}
                    f_{b,c}^n(\bot)(c(s)) & \text{if $b(s)=true$} \\
                    s & \text{if $b(s)=false$}.
                \end{cases}
            \]
            Or, unfolding the definition of $f_{b,c}^n(\bot)$,
            \[
                f_{b,c}(f_{b,c}^n(\bot)) = \lambda s\in State. \begin{cases}
                    c^k(c(s)) & \text{if $b(s)=true$ and $0\leq k <n$} \\
                    & \text{is such that $b(c^i(c(s)))=true$} \\
                    & \text{for all $0\leq i<k$ and} \\
                    & \text{$b(c^k(c(s)))=false$} \\[2ex]
                    \text{undefined} & \text{if $b(s)=true$ and} \\
                    & \text{$b(c^i(c(s)))=true$ for all $i\geq 0$} \\[2ex]
                    c^0(s) & \text{if $b(s)=false$}
                \end{cases}
            \]
            which can be massaged into the definition in the problem statement
            by folding the additional calls to $c$ into the ranges defined in
            the conditions and then combining the top and bottom cases.
        \item If we expand out the function $f_{b,c}(w_{b,c})$, we get the
            definition
            \[
                f_{b,c}(w_{b,c})=\lambda s\in State.\begin{cases}
                    s & \text{if $b(c(s))=false$} \\[2ex]
                    c^k(c(s)) & \text{if $b(s)=true$ and if $k\geq0$} \\
                    & \text{is such that $b(c^i(c(s))=true$ for} \\
                    & \text{all $0\leq i<k$ and $b(c^k(s))=false$} \\[2ex]
                    \text{undefined} & \text{if $b(s)=true$ and} \\
                    & \text{if $b(c^i(c(s))=true$ for all $i\geq 0$}
                \end{cases}
            \]
            which can be massaged into the definition of $w_{b,c}$ similarly to
            in part a). Note that this is $\lim_{n\to\infty}f^n_{b,c}(\bot)$.
        \item If we plug $b$ and $c$ into the definition of $w_{b,c}$ above, we
            see that the top branch never triggers because $b(c^k(s))=true$ will
            never equal $false$. Thus, the least fixed point is $\bot$. Because
            $\bot$ has the empty set as its graph and thus approximates any
            function, we would expect that any function satisfies the fixpoint
            equation. Indeed, per Slide 11,
            \[
            f_{\llbracket \textbf{true}\rrbracket,\llbracket
            \textbf{skip}\rrbracket}=\textit{if}\,(true,w(s),s),
            \]
            so a solution to $w=f_{\llbracket \textbf{true}\rrbracket,\llbracket
            \textbf{skip}\rrbracket}(w)$ is any function such that $w=w$.
    \end{enumerate}
\end{proof}

\begin{problem}
    Show that the relation $\sqsubseteq$ is a partial order with least
    element $\bot$.
\end{problem}
\begin{proof}[Solution]
    This follows trivially from the fact that $\subseteq$ is a partial order
    with least element $\emptyset$, and that two functions are equal if their
    graphs are equal.
\end{proof}

\begin{problem}
    Prove that
    \[
        w=f_{b,c}(w) \Longrightarrow w_{b,c}\sqsubseteq w
    \]
    for all $w\in (State\pf State)$.
\end{problem}
\begin{proof}[Solution]
    We wish to prove that, for any $s,s'\in State$, if $w_{b,c}(s)=s'$ then
    $w(s)=s'$. From the definition of $w_{b,c}$ in Problem 1.1, we see that if
    $w_{b,c}(s)=s'$ then $s'=c^k(s)$ for some $k\geq 0$ and $b(c^i(s))=true$ for
    all $0\leq i<k$ and $b(c^k(s))=false$. We proceed by induction on $k$. If
    $k=0$, then $b(s)=false$ and so $w=\textit{if}\,(false,w(c(s)),s)=c^0(s)$.
    If $k=n+1$, we see that $w=\textit{if}\,(true,w(c(s)),s)=w(c(s))$, and by
    the inductive hypothesis this must equal $c^n(c(s))=c^{n+1}(s)$.
\end{proof}

\section{Least Fixed Points}

\begin{problem}
    Verify the claims implicit in Slide 24: that the relation $\sqsubseteq$
    defined there is a partial order; that $f$ is indeed the lub of the chain
    $f_0\sqsubseteq f_1\sqsubseteq \dots$; and that the totally undefind partial
    function is the least element.
\end{problem}
\begin{proof}[Solution]
        We proved that $\sqsubseteq$ is a partial order in Problem 1.2. By
        construction, every element of the chain $f_i\sqsubseteq f$, so $f$ is
        clearly an upper bound of the chain. Given any other upper bound $g$,
        the domain of $g$ must be at least the union of the domains of elements
        $f_i$, so $f\sqsubseteq g$.
\end{proof}

\begin{problem}
    Prove the claims in Slides 25 and 27.
\end{problem}
\begin{proof}[Solution]
    Let $D$ be a cpo. First, we prove that for $d\in D,\bigsqcup_n d = d$. By
    reflexivity, $d\sqsubseteq d$, so $d$ is an upper bound of the chain. For
    any other upper bound $d'$, by definition we must have $d\sqsubseteq d'$, so
    $d$ is less than all other upper bounds of the chain. Thus, $d$ is the least
    upper bound.

    \noindent
    Next, we prove that for every chain $d_0\sqsubseteq
    d_1\sqsubseteq\dots\sqsubseteq d_n\sqsubseteq\dots$ in $D$,
    \[
        \bigsqcup_i d_i=\bigsqcup_i d_{i+N}
    \]
    for all $N\in \mathbb{N}$. Let $d$ be $\bigsqcup_i d_i$. Every element of
    the righthand chain is also an element of the lefthand chain because it just
    skips the first $N$ elements, so $d$ must be an upper bound of the righthand
    chain. For any other upper bound $d'$, we must also have that $d'$ is an
    upper bound of the lefthand chain because $d_{i+N}\sqsubseteq d'$ for all
    $i$, and $d_j\sqsubseteq d_N$ for all $j<N$. Thus, by the definition of a
    least upper bound, $d\sqsubseteq d'$, so $d$ is also a least upper bound of
    the righthand chain.

    Finally, we prove that for every pair of chains $d_0\sqsubseteq
    d_1\sqsubseteq\dots\sqsubseteq d_n\sqsubseteq\dots$ and $e_0\sqsubseteq
    e_1\sqsubseteq\dots\sqsubseteq e_n\sqsubseteq\dots$ in $D$, if
    $d_n\sqsubseteq e_n$ for all $n\in \mathbb{N}$ then $\bigsqcup_n
    d_n\sqsubseteq\bigsqcup_n e_n$. First, we show that $\bigsqcup_n e_n$ is an
    upper bound of the chain of $d$s. By transitivity of the partial order,
    because $e_i\sqsubseteq\bigsqcup_n e_n$ and $d_i\sqsubseteq e_i$ for all
    $i$, we have that $\bigsqcup_n e_n$ is an upper bound of the $d$ chain.
    Then, by definition of the least upper bound, $\bigsqcup_n
    d_n\sqsubseteq\bigsqcup_n e_n$.
\end{proof}

\begin{problem}
    Verify the claim made in Example 2.3.9 that $f_{b,c}$ is continuous. When is
    it strict?
\end{problem}
\begin{proof}[Solution]
    First, we prove that $f_{b,c}$ is monotone. Let $w_1$ and $w_1$ be partial
    functions $State\pf State$ such that $w_1\sqsubseteq w_2$. We wish to show
    that $\lambda s\in State.\textit{if}\,(b(s), w_1(c(s)), s)\sqsubseteq\lambda
    s\in State.\textit{if}\,(b(s),w_2(c(s)), s)$. This is trivial by case
    analysis on $b(s)$. If $b(s)=true$, then the problem reduces to verifying
    $w_1(c(s))\sqsubseteq w_2(c(s))$, which is true by our choice of $w_1$ and
    $w_2$. If $b(s)=false$, then clearly $s\sqsubseteq s$.

    Next, to show that $f_{b,c}$ is continuous it suffices to show that for each
    chain $d_0\sqsubseteq d_1\sqsubseteq\dots$ in $D$,
    \[
        f_{b,c}\left(\bigsqcup_n d_n\right)\sqsubseteq\bigsqcup_n f_{b,c}(d_n).
    \]
    which is true if you think aabout it.

    It is strict if $b(s)$ is always true.
\end{proof}

\section{Constructions on Domains}
\begin{problem}
    Verify that the constructions given on Slide 32, in Definition 3.2.3, and on
    Slides 35 and 31 do give CPOs and domains. Give the proofs of Propositions
    3.1.1 and 3.2.2.
\end{problem}
\begin{proof}[Solution]
    First, we verify the construction of a product CPO. Let $D_1$ and $D_2$ be
    CPOs and their product is the set $D_1\times D_2$ with partial order 
    \[
        (d_1,d_2)\sqsubseteq(d_1',d_2')
        \stackrel{\text{def}}{\Leftrightarrow}d_1\sqsubseteq d_1' \,\&\,
        d_2\sqsubseteq d_2'.
    \]
    First, we verify that this is a partial order. It is reflexive because for
    any $(d_1,d_2)\in D_1\times D_2$, we have $d_1\sqsubseteq d_1$ and
    $d_2\sqsubseteq d_2$ by the reflexivity of the partial orders on $D_1$ and
    $D_2$, so $(d_1,d_2)\sqsubseteq(d_1,d_2)$. Next we verify antisymmetry. For
    any two elements $(d_1,d_2),(d_1',d_2')\in D_1\times D_2$, if
    $(d_1,d_2)\sqsubseteq (d_1',d_2')$ and $(d_1',d_2')\sqsubseteq (d_1,d_2)$,
    then by definition of the partial order, $d_1\sqsubseteq d_1'$ and
    $d_1'\sqsubseteq d_1$, so by antisymmetry of the partial order on $D_1$, we
    must have $d_1=d_1'$. Similarly, we have $d_2=d_2'$, so
    $(d_1,d_2)=(d_1',d_2')$. Finally, we verify transitivity. If
    $(a_1,a_2)\sqsubseteq (b_1,b_2)$ and $(b_1,b_2)\sqsubseteq (c_1,c_2)$ then
    by transitivity of the underlying partial orders, $a_1\sqsubseteq c_1$ and
    $a_2\sqsubseteq c_2$; therefore, $(a_1,a_2)\sqsubseteq (c_1,c_2)$.

    Next, we verify that $\bigsqcup_{n\geq 0}(d_{1,n},d_{2,n})
    = \left(\bigsqcup_{i\geq 0}d_{1,i}, \bigsqcup_{j\geq 0}d_{2,j}\right)$. Let
    $d_1$ and $d_2$ by te component-wise suprema. By definition of the partial
    order, $(d_{1,n},d_{2,n})\sqsubseteq (d_1,d_2)$ for all $n\geq 0$ so it is
    indeed an upper bound. Given any other upper bound $(d_1',d_2')$, we know by
    the fact that $d_1$ and $d_2$ are suprema that $d_1\sqsubseteq d_1'$ and
    $d_2\sqsubseteq d_2'$, so $(d_1,d_2)$ is the least upper bound. Thus,
    because $D_1$ and $D_2$ are CPOs and have least upper bounds on all their
    chains, $D_1\times D_2$ must also have a least upper bound for each of its
    chains, constructed component-wise as above. Finally, we verify that the
    bottom element is constructed component-wise which is like the same thing as
    everything else so.

    Next, we verify dependent products. It is again the same proof as above---by
    induction, take one product at a time and note that it is a domain, so its
    product with another domain is again a domain.

    I'm tired of this problem.
\end{proof}

\begin{problem}
    Let $X$ and $Y$ be sets and $X_\bot$ and $Y_\bot$ the corresponding flat
    domains. Show that a function $f : X_\bot\rightarrow Y_\bot$ is continuous
    if and only if one of (a) or (b) holds:
    \begin{enumerate}[label=(\alph*)]
        \item $f$ is strict
        \item $f$ is constant.
    \end{enumerate}
\end{problem}
\begin{proof}[Solution]
    In the forward direction, first suppose $f$ is strict. Then if we have two
    elements $x_1,x_2\in X_\bot$ such that $x_1\sqsubseteq x_2$, we know that
    either $x_1=x_2$ or $x_1=\bot_X$. If $x_1=x_2$, then clearly $f(x_1)=f(x_2)$
    so $f(x_1)\sqsubseteq f(x_2)$. If $x_1=\bot_X$, then $f(x_1)=\bot_Y$ because
    $f$ is strict, so by the definition of a least element, $f(x_1)\sqsubseteq
    f(x_2)$. Next, we need to prove that $f$ preserves the suprema of chains.
    Looking at the definition of the partial order for a flat domain, we see
    that all chains must consist of zero or more instances of $\bot$ followed by
    zero or more instances of the same element. Because it is strict, applying
    $f$ to the individual elements of the chain will preserve this structure,
    thus preserving the supremum.

    Otherwise, suppose $f$ is constant and maps everything to some $y\in
    Y_\bot$. By reflexivity of the partial order, it must be monotone, and
    similarly it must preserve suprema of chains because it maps everything to
    $y$ and the supremum of a constaint chain $y\sqsubseteq y\sqsubseteq\dots$
    is $y$.

    In the reverse direction, suppose $f$ is continuous. Then it is monotone, so
    $f(\bot_X)\sqsubseteq f(x)$ for all $x\in X_\bot$. By the definition of the
    partial order, this only holds if $f(\bot_X)=\bot_Y$ ($f$ is strict) or if
    $f(\bot_X)=f(x)$ for all $x$, which is only possible if $f$ is constant.
\end{proof}

\begin{problem}
    Let $\{\top\}$ be a one-element set and $\{\top\}_\bot$ the corresponding
    flat domain. Let $\Omega$ be the domain of `vertical natural numbers.' Show
    that the function domain $\Omega\rightarrow\{\top\}_\bot$ is in bijection
    with $\Omega$.
\end{problem}
\begin{proof}[Solution]
    The underlying set of the function domain $\Omega\rightarrow\{\top\}_\bot$
    consists of all continuous functions $\Omega\rightarrow\{\top\}_\bot$.
    Because $\{\top\}_\bot=\{\top,\bot\}$ is a two-element set and continuous
    functions must be monotone, we can think of these as functions that pick an
    element of $\Omega$ and send everything above that element to $\top$ and
    everything below to $\bot$ (hence the bijection). These functions are
    clearly monotone, and functions that do not fit that description are clearly
    not monotone. The only sticking point is that we are not sure whether the
    selected element should be sent to $\top$ or $\bot$, where either selection
    seems to leave out one function (either the constant function that maps to
    $\bot$ or the constant function that maps to $\top$). However, I think there
    is some subtlety with the natural numbers that I don't yet know, like we can
    think of the spaces between numbers, of which there are contably many, so
    the bijection is still there? It's a Hilbert's Hotel situation.
\end{proof}

\begin{problem}
    Prove the Proposition on Slide 37.
\end{problem}
\begin{proof}[Solution]
    The proposition states that for CPOs $D$, $E$, and $F$, the composision
    function $\circ : ((E\f F)\times (D\f E))\longrightarrow (D\f F)$ defined by
    $g\circ f = \lambda d\in D.g(f(d))$ is continuous. We first prove
    monotonicity. For any $d_1,d_2\in D$ such that $d_1\sqsubseteq d_2$, we
    desire that $(g\circ f)(d_1)\sqsubseteq (g\circ f)(d_2)$ for all $g\in (E\f
    F)$ and $f\in (D\f E)$. By the continuity of $f$, we know that
    $f(d_1)\sqsubseteq f(d_2)$, so it follows by the continuity of $g$ that
    $g(f(d_1))\sqsubseteq g(f(d_2))$.

    Next we prove that suprema of chains are preserved by the composition
    function. For all chains $f(d_1)\sqsubseteq f(d_2)\sqsubseteq\dots\in E$,
    \[
        \bigsqcup_{n\geq 0}g(f(d_n)) = g(\bigsqcup_{n\geq 0}f(d_n))
        =g (f (\bigsqcup_{n\geq 0}d_n)).
    \]
    Thus, the composition of two continuous functions is continuous.
\end{proof}

\section{Scott Induction}

\begin{problem}
    Answer all the ``Why?''s above in the building of chain-closed subsets.
\end{problem}
\begin{proof}[Solution]
    Let $D$ be a CPO. First, we verify that the subset $\{(x,y)\in D\times D\mid
    x\sqsubseteq y\}$ is chain-closed. For any chain
    $(d_1,d_1')\sqsubseteq(d_2,d_2')\sqsubseteq\dots$ such that $d_n\sqsubseteq
    d_n'$ for all $n$, we can take the component-wise suprema so that the least
    upper bound is $(\bigsqcup_n d_n,\bigsqcup_n d_n')$. Because every element
    of the right-projection chain is component-wise greater than the
    left-projection, $\bigsqcup_n d_n'$ must also be an upper bound of the
    left-projection chain. Then by the definition of least upper bound,
    $\bigsqcup_n d_n\sqsubseteq\bigsqcup_n d_n'$, so the least upper bound is in
    the subset.

    It is a more straightforward verification that the subset $\{(x,y)\in
    D\times D\mid x=y\}$ is chain-closed. Clearly, the component-wise least
    upper bounds must be equal because the both projections are equal, so the
    least upper bound is in the subset.

    Next, we verify that for all continuous functions $f:D\f E$, CPOs $D$ and
    $E$, and chain-closed subsets $S\subseteq E$,
    \[
        f^{-1}S=\{x\in D\mid f(x)\in S\}
    \]
    is a chain-closed subset of $D$. Because $S$ is chain-closed, the image of
    the least upper bound of every chain in $f^{-1}S$ under $f$ lies in $S$.
    Then because $f$ is continuous, it preserves least upper bounds, so the
    reverse image of the least upper bound of the chain in $S$ must be the least
    upper bound of the chain in $D$. Thus, it lies in the subset.

    Next, let $S$ and $T$ be chain-closed subsets of the CPO $D$. We verify that
    their union and intersection are chain-closed subsets. For the intersection,
    this is trivial because every chain in $S\cap T$ is also a chain in $S$ and
    $T$, so its least upper bound is in $S$ and $T$. We observe that any chain
    in $S\cup T$ can have any elements not in $S$ removed so that it becomes a
    chain in $S$, and likewise for $T$. The least upper bound of the former
    chain lies in $S$, and the latter in $T$. The larger of these two least
    upper bounds can be taken as the least upper bound of the entire chain,
    which thus lies in $S\cup T$. Finally, we verify that the intersection of a
    family of chain-closed subsets of $D$ is chain-closed, which follows
    trivially from the proof that the intersection of two subsets is
    chain-closed if you don't care about infinity.
\end{proof}

\begin{problem}
    Give an example of a subset $S\subseteq D\times D'$ of a product CPO that is
    not chain-closed, but which satisfies both of the following:
    \begin{enumerate}[label=(\alph*)]
        \item for all $d\in D$, $\{d'\mid (d,d')\in S\}$ is  a chain-closed
            subset of $D'$; and
        \item for all $d'\in D'$, $\{d\mid (d,d')\in S\}$ is  a chain-closed
            subset of $D$.
    \end{enumerate}
\end{problem}
\begin{proof}[Solution]
    TODO: i dont know
\end{proof}

\section{PCF}
\begin{problem}
    Carry out the suggested proof of Proposition 5.4.1.
\end{problem}
\begin{proof}[Solution]
    The proposition states that evaluation in PCF is deterministic: if both
    $M\Downarrow_\tau V$ and $M\Downarrow_\tau V'$, then $V=V'$. The proof is by
    induction on the evaluation rules. For the $\ev{val}$ case, the proposition
    holds by reflexivity.

    For $\ev{succ}$, if $\textbf{succ}(M)\evt{nat} \textbf{succ}(V)$ and
    $\textbf{succ}(M)\evt{nat} \textbf{succ}(V')$, we have by the induction
    hypothesis that $V=V'$. Then by functional equality,
    $\textbf{succ}(V)=\textbf{succ}(V')$.

    For $\ev{pred}$, if $\textbf{pred}(M)\evt{nat}V$ and $\textbf{pred}(M)
    \evt{nat} V'$, we have by the induction hypothesis that $\textbf{succ}(V) =
    \textbf{succ}(V')$. Because \textbf{succ} is injective, this implies that
    $V=V'$.

    For $\ev{zero1}$ and $\evt{zero2}$, we see by the induction hypothesis that
    $M$ evaluates to either \textbf{0} or $\textbf{succ}(V)$ but not both, so
    $\textbf{zero}(M)$ evaluates to either \textbf{true} or \textbf{false} but
    not both.

    Similarly for $\ev{if1}$ and $\ev{if2}$, the condition is either
    \textbf{true} or \textbf{false} but not both, and the triggered branch is
    deterministic, so the if statement is deterministic.

    For $\beta$-reduction, we have determinism of substitution (modulo
    $\alpha$-equivalence). Finally, $\ev{fix}$ uses determinism of
    $\beta$-reduction and substitution.
\end{proof}

\begin{problem}
    Recall that Church's fixpoint combinator in the untyped lambda calculus is
    $Y\stackrel{\text{def}}{=}\lambda f.(\lambda x.f (x\,x))(\lambda
    x.f(x\,x))$. Show that there are no PCF types $\tau_1$, $\tau_2$, $\tau_3$
    so that the typing relation
    \[
        \emptyset\vdash
        \textbf{fn}\,f:\tau_1.(\textbf{fn}\,x:\tau_2.f(x\,x))(\textbf{fn}\,x:\tau_2.f(x\,x)):\tau_3
    \]
    is provable from the axioms and rules in Figure 2.
\end{problem}
\begin{proof}[Solution]
    Suppose for the sake of contradiction that there do exist $\tau_1$,
    $\tau_2$, and $\tau_3$ such that the typing relation holds. By application
    of the $:_\text{fn}$ rule, we see that $\tau_3=\tau_1\rightarrow\tau$ for
    some $\tau$ and $f\mapsto\tau_1\vdash
    (\textbf{fn}\,x:\tau_2.f(x\,x))(\textbf{fn}\,x:\tau_2.f(x\,x)) : \tau$.
    Next, by the $:_\text{app}$ rule, we get that $f\mapsto\tau_1\vdash
    \textbf{fn}\,x:\tau_2.f(x\,x):\tau_2\f\tau$ and $f\mapsto\tau_1\vdash
    \textbf{fn}\,x:\tau_2.f(x\,x):\tau$. However, both premises use the same
    term, so Proposition 5.3.1(i) tells us that $\tau_2\f\tau=\tau$, which is
    impossible.
\end{proof}

\begin{problem}
    Define the following PCF terms:
    \begin{align*}
        plus &\stackrel{\text{def}}{=} \textbf{fix}(\textbf{fn}\,p:nat\f(nat\f
        nat). \textbf{fn}\,x : nat. \textbf{fn}\,y:nat. \\
        & \quad\quad \textbf{if zero}(y) \textbf{ then }x\textbf{ else
        succ}(p\,x\,\textbf{pred}(y))) \\[2ex]
        times &\stackrel{\text{def}}{=} \textbf{fix}(\textbf{fn}\,t:nat\f(nat\f
        nat). \textbf{fn}\,x : nat. \textbf{fn}\,y:nat. \\
        & \quad\quad \textbf{if zero}(y) \textbf{ then 0 else
        }plus(t\,x\,\textbf{pred}(y))\,x) \\[2ex]
        fact &\stackrel{\text{def}}{=} \textbf{fix}(\textbf{fn}\,f:nat\f(nat\f
        nat). \textbf{fn}\,x : nat. \\
        & \quad\quad \textbf{if zero}(x) \textbf{ then succ(0) else }times\,
        x(f\,\textbf{pred}(x))).
    \end{align*}
    Show by induction on $n\in \mathbb{N}$ that for all $m\in \mathbb{N}$
    \begin{align*}
        plus\,\textbf{succ}^m(\textbf{0})\,\textbf{succ}^n(\textbf{0})
        &\Downarrow_{nat} \textbf{succ}^{m+n}(\textbf{0}) \\
        times\,\textbf{succ}^m(\textbf{0})\,\textbf{succ}^n(\textbf{0})
        &\Downarrow_{nat} \textbf{succ}^{m*n}(\textbf{0}) \\
        fact\,\textbf{succ}^n(\textbf{0}) &\Downarrow_{nat}
        \textbf{succ}^{n!}(\textbf{0})
    \end{align*}
\end{problem}
\begin{proof}[Solution]
    For the base case, we see that
    \begin{align*}
        plus\,\textbf{succ}^m(\textbf{0})\,\textbf{0} &\evt{nat} \textbf{if
        zero}(\textbf{0})\textbf{ then }\textbf{succ}^m(\textbf{0}) \\
        &\quad\quad \textbf{ else
        }\textbf{succ}(plus\,\textbf{succ}^m(\textbf{0)}\,\textbf{pred}(\textbf{0}))
        \\ &= \textbf{succ}^{m+0}(\textbf{0}),
    \end{align*}
    so the statement holds. For the inductive case,
    \begin{align*}
        plus\,\textbf{succ}^m(\textbf{0})\,\textbf{succ}^n(\textbf{0})
        &\evt{nat} \textbf{if zero}(\textbf{succ}^n(\textbf{0)})\textbf{ then
        }\textbf{succ}^m(\textbf{0}) \\
        &\quad\quad \textbf{ else
        }\textbf{succ}(plus\,\textbf{succ}^m(\textbf{0})\,\textbf{pred}(\textbf{succ}^n(\textbf{0}))
        \\
        &= \textbf{succ}(plus\,\textbf{succ}^m(\textbf{0})\,\textbf{succ}^{n-1}(\textbf{0})) \\
        &= \textbf{succ}(\textbf{succ}^{m+n-1}(\textbf{0})) =
        \textbf{succ}^{m+n}(\textbf{0}).
    \end{align*}
    You get the idea.
\end{proof}

\section{Denotational Semantics of PCF}
\begin{problem}
    Suppose
    \begin{align*}
        \Gamma &\vdash M:\tau \\
        \Gamma[x\mapsto\tau] &\vdash M':\tau'
    \end{align*}
    (so that by Proposition 5.3.1(ii) we also have $\Gamma\vdash M'[M/x] :
    \tau'$). Then for all $\rho\in\den{\Gamma}$
    \[
        \den{\Gamma\vdash M'[M/x]}(\rho) = \den{\Gamma[x\mapsto\tau] \vdash
        M'}(\rho[x\mapsto\den{\Gamma\vdash M}]).
    \]
\end{problem}
\begin{proof}[Solution]
    We proceed by induction on $M'$. If $M'$ is a constant (\textbf{0},
    \textbf{true}, or \textbf{false}), then $x$ does not appear free in $M'$, so
    the statement holds trivially. If $M'=x$, then
    \[
        \den{\Gamma\vdash x[M/x]}(\rho) = \den{\Gamma\vdash M}(\rho) =
        \den{\Gamma[x\mapsto\tau] \vdash x}(\rho[x\mapsto\den{\Gamma\vdash M}]).
    \]
    If $M'=\textbf{succ}(N)$, then
    \begin{align*}
        &\den{\Gamma\vdash \textbf{succ}(N)[M/x]}(\rho) = \den{\Gamma\vdash
        \textbf{succ}(N[M/x])}(\rho) \\
        &= \begin{cases}
            \den{\Gamma\vdash N[M/x]}(\rho) + 1 & \text{if }
            \den{\Gamma\vdash N[M/x]}(\rho) \neq\bot \\
            \bot & \text{if }\den{\Gamma\vdash N[M/x]}(\rho) = \bot.
        \end{cases} \\
        &= \begin{cases}
            \den{\Gamma[x\mapsto\tau]\vdash N}(\rho[x\mapsto\den{\Gamma\vdash
            M}]) + 1 & \text{if } \den{\Gamma[x\mapsto\tau]\vdash
            N}(\rho[x\mapsto\den{\Gamma\vdash M}]) \neq\bot \\
            \bot & \text{if }\den{\Gamma[x\mapsto\tau]\vdash
            N}(\rho[x\mapsto\den{\Gamma \vdash M}]) = \bot
        \end{cases} \\
        &= \den{\Gamma[x\mapsto\tau]\vdash \textbf{succ}(N)}(\rho
        [x\mapsto\den{\Gamma\vdash M}).
    \end{align*}
    Using the same logic of passing the substitution into the operation, we can
    see that the same holds for \textbf{pred}, \textbf{zero}, and \textbf{if}
    (using the inductive hypothesis on $M_1$ so see that the same branch of the
    if statement is triggered and then the inductive hypothesis on that branch
    to see that they evaluate to the same thing). If $M'=M_1 M_2$,
    \begin{align*}
        &\den{\Gamma\vdash (M_1 M_2)[M/x]}(\rho) \\
        &= \den{\Gamma\vdash (M_1[M/x])(M_2[M/x])}(\rho) \\
        &= (\den{\Gamma\vdash M_1[M/x]}(\rho))(\den{\Gamma\vdash M_2[M/x]}(\rho)) \\
        &= (\den{\Gamma[x\mapsto\tau] \vdash M_1}(\rho[x\mapsto\den{\Gamma\vdash M}]))
        (\den{\Gamma[x\mapsto\tau] \vdash M_2}(\rho[x\mapsto\den{\Gamma\vdash M}])) \\
        &= \den{\Gamma[x\mapsto\tau] \vdash M_1 M_2}(\rho[x\mapsto\den{\Gamma\vdash M}]).
    \end{align*}
    If $M'=\textbf{fn }y:\tau'.N$ where $y\neq x$ and $y\notin dom(\Gamma)$, then
    \begin{align*}
        &\den{\Gamma\vdash(\textbf{fn }y:\tau'.N)[M/x]}(\rho) \\
        &= \lambda d\in\den{\tau'}.\den{\Gamma[y\mapsto\tau']\vdash
        N[M/x]}(\rho[y\mapsto d]) \\
        &= \lambda d\in\den{\tau'}.\den{\Gamma[y\mapsto\tau'][x\mapsto\tau]
        \vdash N}(\rho[y\mapsto d][x\mapsto \den{\Gamma[y\mapsto\tau']\vdash
        M}]) \\
        &= \den{\Gamma[x\mapsto\tau]\vdash \textbf{fn }y:\tau'.N}
        (\rho[x\mapsto\den{\Gamma\vdash M}])
    \end{align*}
    by the initial assumption that $y$ does not appear free in $\Gamma$ and
    probably some theorem we have that you can remove a substitution when it
    doesn't appear free. Finally, if $M'=\textbf{fix}(N)$,
    \begin{align*}
        \den{\Gamma\vdash \textbf{fix}(N)[M/x]}(\rho) &=
        \textit{fix}(\den{\Gamma\vdash N[M/x]}(\rho)) \\
        &= \textit{fix}(\den{\Gamma[x\mapsto\tau]\vdash N})(\rho[x\mapsto
        \den{\Gamma\vdash M}]) \\
        &= \den{\Gamma[x\mapsto\tau]\vdash \textbf{fix}(N)}(\rho[x\mapsto
        \den{\Gamma\vdash M}]).
    \end{align*}
    Phew.
\end{proof}

\begin{problem}
    Defining $\Omega_\tau\stackrel{\text{def}}{=}\textbf{fix(fn }x : \tau.x)$,
    show that $\den{\Omega_\tau}$ is the least element $\bot$ of the domain
    $\den{\tau}$. Deduce that $\den{\textbf{fn }x:\tau.\Omega_\tau} =
    \den{\Omega_{\tau\f\tau}}$.
\end{problem}
\begin{proof}[Solution]
    By unfolding definitions, we see that
    % \begin{align*}
    %     \den{\Omega_\tau} &= \den{\emptyset\vdash
    %     \textbf{fix}(\textbf{fn }x:\tau\,.\,x)}(\bot) \\
    %     &= \textit{fix}(\den{\emptyset\vdash \textbf{fn }x:\tau\,.\,x}(\bot)) \\
    %     &= \textit{fix}(\lambda d\in\den{\tau}.\den{\emptyset[x\mapsto\tau]
    %     \vdash x}(\bot[x\mapsto d])) \\
    %     &= \textit{fix}(\lambda d\in\den{\tau}.d) \\
    %     &= (\lambda d\in\den{\tau}.d)\den{\Omega_\tau} = \den{\Omega_\tau},
    % \end{align*}
    % so $\den{\Omega=\tau}$ is the least function that is equal to itself. This
    % gives us the intuition that $\den{\Omega_\tau}=\bot$.
    \[
        \den{\Omega_\tau} = \den{\emptyset\vdash \textbf{fix}(\textbf{fn
        }x:\tau\,.\,x)}(\bot) = \textit{fix}(\lambda d\in\den{\tau}.d).
    \]
    By Tarski's Fixed Point Theorem,
    \[
        \textit{fix}(\lambda d\in\den{\tau}.d) = \bigsqcup_{n\geq 0}(\lambda
        d\in\den{\tau}.d)^n(\bot) = \bigsqcup_{n\geq 0}\bot = \bot.
    \]
    Because denotation is a congruence, we can rewrite $\den{\Omega_\tau}=\bot$
    so that
    \[
        \den{\textbf{fn }x:\tau\,.\,\Omega_\tau} = \den{\emptyset\vdash \textbf{fn
        }x:\tau\,.\,\bot}(\bot) = \lambda d\in\den{\tau}.\bot.
    \]
    Because it is undefined everywhere, it must be the least element of
    $\den{\tau\f\tau}$ and thus it is $\den{\Omega_{\tau\f\tau}}$.
\end{proof}

\section{Relating Denotational and Operational Semantics}

\begin{problem}
    For any PCF type $\tau$ and any closed terms $M_1,M_2\in\text{PCF}_\tau$,
    show that
    \[
        \forall V:\tau\, (M_1\Downarrow_\tau V \Leftrightarrow M_2\Downarrow_\tau V)
        \Rightarrow M_1 \cong_\text{ctx} M_2 : \tau.
    \]
\end{problem}
\begin{proof}[Solution]
    To prove that $M_1\cong_\text{ctx}M_2 : \tau$, it suffices to show that
    $M_1\leq_\text{ctx}M_2:\tau$ and $M_2\leq_\text{ctx}M_1:\tau$, and by the
    Proposition on Slide 68 it suffices to show $\den{M_1} \lhd_\tau M_2$ and
    $\den{M_2}\lhd_\tau M_1$. Both of these are discharged by applying Lemma
    7.2.1(iii), instantiating $d_1=d_2=\den{M_1}$ and $d_1=d_2=\den{M_2}$ for
      the second. By reflexivity, we have that $\den{M_1}\sqsubseteq\den{M_1}$
      and by the Fundamental Property we have $\den{M_1}\lhd_\tau M_1$, and by
      assumption we have that $M_1$ and $M_2$ evaluate to the same value.
\end{proof}

\begin{problem}
    Use the statement in 7.1 to show that $\beta$-conversion is valid up to
    contextual equivalence in PCF, in the sense that for all $\textbf{fn
    }x:\tau\,.\,M_1\in\text{PCF}_{\tau\f\tau'}$ and $M_2\in\text{PCF}_\tau$
    \[
        (\textbf{fn }x:\tau\,.\,M_1)M_2\cong_\text{ctx} M_1[M_2/x] : \tau'.
    \]
\end{problem}
\begin{proof}[Solution]
    By the above, it suffices to show that
    \[
        \forall V:\tau'\,((\textbf{fn }x:\tau\,.\,M_1)M_2 \Downarrow_{\tau'} V
        \Leftrightarrow M_1[M_2/x] \Downarrow_{\tau'} V)).
    \]
    We first prove the forward direction by supposing that $(\textbf{fn
    }x:\tau\,.\,M_1)M_2 \Downarrow_{\tau'} V$. By observing our big step
    semantics, the only rule that can apply is $\Downarrow_\text{cbn}$, giving
    us the assumptions $\textbf{fn }x:\tau\,.\,M_1 \Downarrow_{\tau\f\tau'}
    \textbf{fn }x:\tau\,.\,M'$ and $M'[M_2/x] \Downarrow_{\tau'} V$. Because
    functions are already values, we see from the first assumption that
    $M'=M_1$, so the second assumption becomes $M_1[M_2/x]\Downarrow_{\tau'} V$,
    which is exactly what we wished to prove. In the reverse direction, the
    exact same logic applies, allowing us to apply the $\ev{cbn}$ rule.
\end{proof}
\end{document}
